\documentclass[12pt]{article}
\usepackage{CJKutf8}
\usepackage{geometry}
\usepackage{listings}
\usepackage{amsmath}
\usepackage{commath}

\newgeometry{vmargin={6mm,10mm}, hmargin={10mm,10mm}}   
\begin{CJK}{UTF8}{bsmi}
\author{b05502087 王竑睿}
\date{}
\title{數值線性代數 midterm}

\lstset{frame=tb,
    language=Matlab}

\begin{document}
\maketitle
    \section{Given an m-by-m nonsingular matrix A, how do you efficiently solve
    the following problems, using Guassian elimination with partial pivoting (GEPP) ?}
        \subsection*{(a) Solve the linear system $A^kx = b$, where k is a positive number.}
            \subsubsection*{algorithms}
                \begin{equation*}
                    \begin{aligned}
                        &A^kx=b\\
                        \Rightarrow &A(A^{k-1}x)=b\\
                        \Rightarrow &Ay=b  \qquad \text{(令 $y=A^{k-1}x$)}\\
                        &\text{由高斯消去法解得y}\\
                        \Rightarrow &A^{k-1}x=y\\
                        &\text{重複此步驟逐步減少A的次方}\\
                    \end{aligned}    
                \end{equation*}
            \subsubsection*{pseudocode}
                \begin{lstlisting}
                    function xk = powerA(A,pow,b)
                        bk(:) = b(:);
                        bk = bk';
                        for count=pow:-1:1
                            A
                            bk
                            x = gaussianelim(A,bk);
                            A*x-bk
                            if(count~=1)
                                bk=x;
                            end
                        end
                        xk=x
                        %Bk=bk
                        %Ak=A
                    end
                \end{lstlisting}
        \newpage
                \textbf{\centerline{partial pivoting}}
                \begin{lstlisting}
                    function x = gaussianelim(A,b);
                        [row,col]=size(A);
                        n = row;
                        x = zeros(n,1);
                        for k=1:n-1
                            %do partial pivoting
                            for i=k+1:n 
                                if A(i,k) > A(k,k)
                                    A([i, k], :) = A([k, i], :);
                                    b([i, k], :) = b([k, i], :);
                                end
                            end
                            for i=k+1:n
                                xMultiplier = A(i,k)/A(k,k);
                                for j=k+1:n
                                    A(i,j) = A(i,j)-xMultiplier*A(k,j);
                                end
                                b(i) = b(i)-xMultiplier*b(k);
                            end
                        end
                        % backsubstitution:
                        x(n) = b(n)/A(n,n);
                        for i=n-1:-1:1
                            summation = b(i);
                            for j=i+1:n
                                summation = summation-A(i,j)*x(j);
                            end
                            x(i) = summation/A(i,i);
                        end
                    end
                \end{lstlisting}
            \subsubsection*{required flops}
                \begin{itemize}
                    \item 高斯消去法使用partial pivoting約有$\dfrac{2}{3}m^3+\dfrac{1}{2}m(m-1)$個flops
                    \item back substitution約有$m^2$個flops
                    \item 做k次會需要$k\times(\dfrac{2}{3}m^3+\dfrac{1}{2}m(m-1)+m^2)$個flops
                \end{itemize}
    \newpage
        \subsection*{(b) Compute $\alpha = c^TA^{-1}b$.}
            \subsubsection*{algorithms}
                \begin{equation*}
                    \begin{aligned}
                        &\alpha=c^TA^{-1}b\\
                        \Rightarrow &\alpha=c^Tx \qquad \text{(其中 $Ax=b$)}\\
                        &\text{由高斯消去法解得x}\\
                        \Rightarrow &\alpha=c^Tx
                    \end{aligned}    
                \end{equation*}
            \subsubsection*{pseudocode}
                \begin{lstlisting}
                    function alpha=cAb(c,A,b)
                        x = gaussianelim(A,b);
                        alpha = c*x;
                    end
                \end{lstlisting}
                \textbf{\centerline{partial pivoting}}
                \begin{lstlisting}
                    function x = gaussianelim(A,b);
                        [row,col]=size(A);
                        n = row;
                        x = zeros(n,1);
                        for k=1:n-1
                            %do partial pivoting
                            for i=k+1:n 
                                if A(i,k) > A(k,k)
                                    A([i, k], :) = A([k, i], :);
                                    b([i, k], :) = b([k, i], :);
                                end
                            end
                            for i=k+1:n
                                xMultiplier = A(i,k)/A(k,k);
                                for j=k+1:n
                                    A(i,j) = A(i,j)-xMultiplier*A(k,j);
                                end
                                b(i) = b(i)-xMultiplier*b(k);
                            end
                        end
                        % backsubstitution:
                        x(n) = b(n)/A(n,n);
                        for i=n-1:-1:1
                            summation = b(i);
                            for j=i+1:n
                                summation = summation-A(i,j)*x(j);
                            end
                            x(i) = summation/A(i,i);
                        end
                    end
                \end{lstlisting}
            \subsubsection*{required flops}
                \begin{itemize}
                    \item 高斯消去法使用partial pivoting得到x約需要$\dfrac{2}{3}m^3+\dfrac{1}{2}m(m-1)$個flops
                    \item back substitution約有$m^2$個flops
                    \item 向量c乘以向量x會需要大約$2m$個flops
                    \item 因此總共約需要$\dfrac{2}{3}m^3+\dfrac{1}{2}m(m-1)+m^2+2m$個flops
                \end{itemize}
        \subsection*{(c) Solve the matrix equation $AX = B$, where B is m-by-n.}
            \subsubsection*{algorithms}
                \begin{equation*}
                    \begin{aligned}
                        &\text{將B視為$b_1,b_2,...,b_n$}\\
                        &\text{將X視為$x_1,x_2,...,x_n$}\\
                        \Rightarrow &\text{$AX=B$可視為n個linear system}\\
                        &\text{($Ax_i=b_i$ for i=1,2,...,n)}\\
                        \Rightarrow &\text{亦可使用高斯消去法一一解出}\\
                    \end{aligned}    
                \end{equation*}
            \subsubsection*{pseudocode}
            \textbf{\centerline{partial pivoting}}
                \begin{lstlisting}
                    function x = gaussianelim(A,B);
                        [row,col]=size(A);
                        [rowB,colB]=size(B);
                        n = row;
                        x = zeros(n,colB);
                        for k=1:n-1
                            %do partial pivoting
                            for i=k+1:n 
                                if A(i,k) > A(k,k)
                                    A([i, k], :) = A([k, i], :);
                                    B([i, k], :) = B([k, i], :);
                                end
                            end
                            for i=k+1:n
                                xMultiplier = A(i,k)/A(k,k);
                                for j=k+1:n
                                    A(i,j) = A(i,j)-xMultiplier*A(k,j);
                                end
                                B(i,:) = B(i,:)-xMultiplier*B(k,:);
                            end
                        end
                        % backsubstitution:
                        for k=1:colB
                            x(n,k) = B(n,k)/A(n,n);
                            for i=n-1:-1:1
                                summation = B(i,k);
                                for j=i+1:n
                                    summation = summation-A(i,j)*x(j,k);
                                end
                                x(i,k) = summation/A(i,i);
                            end
                        end
                    end
                \end{lstlisting}
            \subsubsection*{required flops}
            \begin{itemize}
                \item 高斯消去法使用partial pivoting得到x約需要$\dfrac{2}{3}m^3+m(m-1)n+\dfrac{1}{2}m(m-1)$個flops
                \item back substitution約有$nm^2$個flops (B可視為n個行向量)
                \item 因此總共約需要$\dfrac{2}{3}m^3+m(m-1)n+\dfrac{1}{2}m(m-1)+nm^2$個flops
            \end{itemize}
    \section{The aim of this problem is to generalize the approach described in class
    for solving a tridiagonal system of equations to a pentadiagonal system in x $\in$ $R^m$
    of the form}
        $$a_jx_{j-2} + b_jx_{j-1} + c_jx_j + d_jx_{j+1} + e_jx_{j+2} = f_j$$
        for $j = 1, 2, . . . , m$ with
        \begin{equation*}
            \left[
            \begin{array}{c}
                x_{-1}\\
                x_{0}\\
                x_{m+1}\\
                x_{m+2}\\
            \end{array}
            \right]
            =
            \left[
            \begin{array}{c}
                \alpha_{-1}\\
                \alpha_{0}\\
                \alpha_{m+1}\\
                \alpha_{m+2}\\
            \end{array}
            \right]
        \end{equation*}
    \subsubsection*{algorithms}
        此矩陣形如\\
        \begin{equation*}
            A =
            \left[
                \begin{array}{ccccccccccc}
                    a&b&c&d&e&...&0&0&0&0\\
                    0&a&b&c&d&e&...&0&0&0\\
                    0&0&a&b&c&d&e&...&0&0\\
                    ... \\
                    0&0&0&...&0&a&b&c&d&e\\
                \end{array}
            \right]
        \end{equation*}
        \begin{itemize}
            \item 此linear system有以下限制
                \subitem $x_{-1}=\alpha_{-1}$
                \subitem $x_{0}=\alpha_{0}$ 
                \subitem $x_{m+1}=\alpha_{m+1}$
                \subitem $x_{m+2}=\alpha_{m+2}$\\
                \subitem 但受影響的只有$f_1$ $f_2$ $f_{m-1}$ $f_{m}$
                \subitem 可先將$f_1$ $f_2$ $f_{m-1}$ $f_{m}$
                \subitem 分別減去$a_1\alpha_{-1}+b_1\alpha_{0}$、$a_2\alpha_{0}$ 、$e_{m-1}\alpha_{m+1}$、$d_{m}\alpha_{m+2}+e_{m}\alpha_{m+2}$
            \item 則接下來就可以只考慮
                \subitem $x_1,x_2,...,x_m$
                \subitem $A_{(1,1:m)}$,$A_{(2,1:m)}$,...,$A_{(m,1:m)}$
            \item 利用complete pivoting解$A_{(1:m,1:m)}x_{(1:m)}=f_{(1:m)}$
        \end{itemize}
    \subsubsection*{pseudocode}
        \textbf{\centerline{solve pentadiagonal system}}
        \begin{lstlisting}
            function x = penta(A,myalpha,f)
                %[A,myalpha,f]=getMatrix(m);
                m = size(A,1);
                
                xtemp = zeros(m+4,1);
                xtemp(1) = myalpha(1);
                xtemp(2) = myalpha(2);
                xtemp(m+3) = myalpha(3);
                xtemp(m+4) = myalpha(4);
                f = f - A*xtemp;
                
                [Arow Acol] = size(A);
                Anew = A(:,3:Acol-2);
                %x = Anew\f;
                x = gaussianelimComplete(Anew,f);
                x = [myalpha(1:2);x;myalpha(3:4)];
            end
        \end{lstlisting}
        \textbf{\centerline{Gaussian elimination with complete pivoting}}
        \begin{lstlisting}
            function x = gaussianelimComplete(A,b);
                [row,col]=size(A);
                m = row;
                n = col;
                x = zeros(n,1);
                index = [1:n]';
                for k=1:m-1
                    %do complete pivoting
                    maxRow = -1;
                    maxCol = -1;
                    maxElement = -1;
                    for i=k:m 
                        for j=k:n
                            if abs(A(i,j)) > maxElement
                                %row change
                                maxRow = i;
                                %column change
                                maxCol = j;
                                %Element
                                maxElement = abs(A(i,j));
                            end
                        end
                    end
                    A([maxRow,k],:) = A([k,maxRow],:);
                    b([maxRow,k],:) = b([k,maxRow],:);
                    A(:,[maxCol,k]) = A(:,[k,maxCol]);
                    index([maxCol,k],:) = index([k,maxCol],:);
                    
                    for i=k+1:m
                        xMultiplier = A(i,k)/A(k,k);
                        A(i,k:n) = A(i,k:n) - xMultiplier*A(k,k:n);
                        b(i) = b(i)-xMultiplier*b(k);
                    end
                end
                % backsubstitution:
                x(n) = b(n)/A(n,n);
                for i=n-1:-1:1
                    summation = b(i);
                    for j=i+1:n
                        summation = summation-A(i,j)*x(j);
                    end
                    x(i) = summation/A(i,i);
                end
                %At = A;
                %bt = b;
                toSort = [x,index];
                AfterSort = sortrows(toSort,2);
                x = AfterSort(:,1);
            end
        \end{lstlisting}
    \subsubsection*{required flops}
        \begin{itemize}
            \item 處理$f_1$ $f_2$ $f_{m-1}$ $f_{m}$約需要12個flop
            \item 高斯消去法使用complete pivoting得到x約需要$\dfrac{2}{3}m^3+\dfrac{1}{3}m^3$個flops
            \item back substitution約有$m^2$個flops
            \item 因此總共約需要$m^3+m^2+12$個flops
        \end{itemize}
\end{CJK}
\end{document}